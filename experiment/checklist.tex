\documentclass{article}

%% Bring items closer together in list environments
% Prevent infinite loops
\let\Itemize =\itemize
\let\Enumerate =\enumerate
\let\Description =\description
% Zero the vertical spacing parameters
\def\Nospacing{\itemsep=0pt\topsep=0pt\partopsep=0pt\parskip=0pt\parsep=0pt}
% Redefine the environments in terms of the original values
\renewenvironment{itemize}{\Itemize\Nospacing}{\endlist}
\renewenvironment{enumerate}{\Enumerate\Nospacing}{\endlist}
\renewenvironment{description}{\Description\Nospacing}{\endlist}

\usepackage[left=1.0in,top=1.0in,right=1.0in,bottom=1.2in,nohead,nofoot]{geometry}
\begin{document}

\begin{center}
\LARGE JavaGrok Experiment Checklist
\end{center}

As you work, we anticipate that you will need to refer to the existing source
code to complete your tasks. When you inspect the source code to answer a
question that arises during development, we ask that you note the way in which
the question was answered (or not). Specifically:

\begin{enumerate}
\item When your question is answered by comments (either inline or Javadoc
  formatted).
\item When your question is answered by an annotation on a method or parameter.
\item When your question is answered by reading the code itself, either the
  bodies of methods or simply their signatures (method names and parameters).
\item When your question is answered in some other way (please write down a
  word or two describing the means by which your question was answered).
\item When your question was not answered by any means.
\end{enumerate}

We have provided a space below for each of these categories.  Any unambiguous
method of counting works fine; we suggest using ticks or hash marks.  Please
tell whoever is helping you through the study when you start working, and when
you finish.  When you are done, there is a short survey about your experiences
and feelings as you completed the test application.
\section*{}

Question answered by:

\subsubsection*{Comments}
\begin{picture}(\textwidth,40)
\put(0,0){\line(0,1){1}}
\end{picture}
\subsubsection*{Annotation}
\subsubsection*{Code}
\subsubsection*{Other means}
\subsubsection*{Not answered}

\begin{tabular*}{\textwidth}{| p{0.174\textwidth} | p{0.174\textwidth} |
    p{0.174\textwidth} | p{0.175\textwidth} | p{0.175\textwidth} | }
  \hline
  \multicolumn{5}{|l|}{Question answered by:} \\ \hline
  comments & annotation & code & other means & not answered \\ \hline
   &  & &  & \\
   &  & &  & \\
   &  & &  & \\
   &  & &  & \\
  \hline
\end{tabular*}

\newpage

\begin{center}
\LARGE JavaGrok User Study Questionnaire
\end{center}

\begin{enumerate}
\item Did you find the experience of using this library to be pleasant, normal,
or frustrating? (circle one)\\
\item Did you find the library documentation to generally be to too detailed, sufficiently
detailed, or not detailed enough? (circle one)\\
\item Did you find the library documentation to be generally very helpful,
somewhat helpful, or often unhelpful?\\
\item Did you find the documentation to be easy to read, of acceptable
readability, or difficult to read?  Note that we are referring to ease of
finding information, not quality of formatting.\\
\item Did you find the documentation to be cluttered, okay, or spread out too
far?\\
\item Was anything documented that you found unnecessary?\\\vspace{1in}
\item Was anything not documented that you wished had been present?\\\vspace{1in}
\item What part of using the library did you find to be the most
frustrating?\\\vspace{1in}
\item What part of using the library did you find to be the least
frustrating?\\\vspace{1in}
\end{enumerate}

\end{document}
