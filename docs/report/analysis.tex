\section{Analyses}

In this section we describe the analyses we perform: their implementation,
output, and why each analysis would be useful for a developer.

\subsection{Javarifier}
\label{sec:Javarifier}

Javarifier is a tool to infer reference immutability information built in
conjuction with research done by Quinonez et al.~\cite{Javarifier}. It infers
mutability constraints for object fields, method arguments and receivers.  We
use \textit{mutate} to mean updating the value of an object's fields, calling a
method on an object that updates the values of its fields, or calling a method
on an object that mutates an object referenced by one of its fields.

Below we enumerate the annotations inferred by Javarifier which are utilized by
JavaGrok, and a summary of what they communicate to the developer:

\texttt{@ReadOnly} When appearing on a method, this indicates to the developer
that the method will not mutate its receiver. When appearing on a method
argument, this indicates to the developer that the method will not mutate the
argument through the supplied reference.

\texttt{@Mutable} When appearing on a method, this indicates to the developer
that the method may mutate its receiver. When appearing on a method argument,
this indicates to the developer that the method may mutate the argument through
the supplied reference.

Javarifier generates two additional annotations which are important for type
checking, but not important when simply communicating method behavior to a
developer.

\texttt{@QReadOnly} appears on type parameters of method arguments, for example
\texttt{List<@QReadOnly Date>}. Such an annotation on a method argument
indicates that the method accepts both \texttt{List<@ReadOnly Date>} and
\texttt{List<@Mutable Date>}. However, the method itself is restricted to only
operations allowed on both read-only and mutable dates, which is strictly a
subset of the operations allowed on read-only dates.

\texttt{@PolyRead} indicates that a method and/or its arguments are polymorphic
over mutability. Such a method may legally be instantiated with
\texttt{@ReadOnly} receiver and/or arguments and thus must not mutate the
receiver or arguments.

We convert both of the above annotations to \texttt{@ReadOnly}, as
that is sufficient to communicate the method's behavior to the developer and we
are not concerned with type-checking mutability.

\subsection{Uno}

Uno~\cite{Uno} infers alias and encapsulation properties. The
tool generates predicates which provide information about how a certain
function treats its parameters, return values and fields when called. More
concretely, it infers whether or not a function captures or leaks a
reference, or returns a new unique reference.

Below we enumarete the annotations we use and which predicates it uses.

\texttt{@UniqueReturn} on method declarations. This annotations indicates 
that the return value is a unique reference to its referent. This maps directly
to Uno's predicate UniqRet to be true.
\texttt{@NonUniqueReturn} can occur on method declarations and means that 
there is at least one possible path between method entrance and exit in
which the returned reference is not the only reference to its referent.
This maps directly to Uno's predicate UniqRet to be false.
\texttt{@Retained} when appearing in front of a method parameter it indicates
that the're exists at least one path between method entrance and exit in 
which the reference get's captured. For example might the called method store
a reference to the referent in a local field.
\texttt{@NotRetained} respectively indicates that the method has no execution 
paths which retains a reference to the passed object.
The retention annotations correspond to Uno's LentPar predicate.

Uno infers a lot of other predicates which at first thought also might be 
interesting for a programmer. But turning them all into annotations will
add to much information to the source code. In a self-experiment we 
realised that the amount of additional information might easily distract
the programmer instead of help. Therefore we only use the predicates 
described above.

The tool generates predicates which are stored in a single separate file. 
To integrate the analysis results of UNO into the Java documentation we
use our own framework. We implemented an AST traversal inside the Java 7 
compiler that, at initialization, reads in the file generated by UNO and 
stores the information in a hashset. During the iteration over the AST
our visitor inserts annotation at the appropriate places.

\subsection{Thrown Exceptions}

Java requires that checked exceptions be declared by methods that raise them,
but frequently the conditions under which exceptions are raised are unclear to a
library user. Even with checked exceptions, often a supertype of a thrown
exception is declared.  While knowing that an \texttt{IOException} might be
thrown is useful, knowing that the method might throw either a
\texttt{FileNotFoundException} or \texttt{InterruptedByTimeoutException} is more
useful to a developer, because knowing the specific exception may help to debug
a program; knowing that a \texttt{FileNotFoundException} was thrown tells a
developer to check that file paths are correct and that any tasks that should
have created the file ran correctly, while an \texttt{IOException} could
indicate a variety of problems with local files, or any of a number of problems
with the network.  Additionally, unchecked exceptions need not be declared in Java method
signatures or documentation and can remain entirely invisible to the library
user until they are encountered at runtime.

To alleviate this, we infer some of the conditions under which a method would
throw an exception, and produce annotations of the form:

\begin{verbatim}
@Throws(when =
    "IllegalArgumentException when (x < 0)")
public Object getElement(int x) { ... }
\end{verbatim}

This inferred information can inform developers of unchecked exceptions,
and alert them to otherwise undocumented assumptions of a library without
requiring developers to dig through the library's code.
Annotations may contain a list of exceptions and conditions, and for some
complex conditions we reduce the condition to ``sometimes'' because our analysis
performs no symbolic execution, and is only based on syntactic structure.  We do not use an
existing analysis tool because none were available for this task, but our
simple analysis provides useful results.

\subsubsection{Implementation}

Our exception analysis is implemented in two phases as plugins to the JDK 7 compiler.
In the first phase, the
abstract syntax tree for every method of the program is traversed in the
following manner.  The analysis walks over every statement of the method,
descending recursively into control flow constructs.  The analysis keeps a stack
of conditional branches, pushing branch conditions when entering the positive
branch of a conditional, and negating the most recent condition when entering
the \texttt{else} clause.  A counter is also kept for how many times the
analysis has entered a more complex control structure, such as a loop or
\texttt{switch} statement whose conditions for execution are more complicated,
and require reasoning about iteration and ``falling through'' cases.  When
analysis of a control structure is complete, a clause is popped from the
branch stack or the counter is decremented as appropriate.

When entering a method or a \texttt{try}/\texttt{catch} construct, we push a new empty set of exceptions onto
a stack that tracks what exceptions are thrown at the current level of
\texttt{try} nesting: the try-stack.
When a throw statement is found, the statement's AST node is added to the
topmost set on the try-stack, 
along with the current branch condition if no complex control structures have
been encountered, otherwise it is added with a flag that the condition was more
complicated than we trace.  When a \texttt{try} body has been traversed, we
remove exceptions from that top set that are subclasses of caught exception
types in the associated \texttt{catch} bodies.  We then merge the topmost set
into the next set on the try-stack, and process the bodies of
the \texttt{catch} clauses.  When traversal of a method is complete, we add the
single remaining set on the try-stack to a hash map from method definition AST
nodes to sets of exception-condition pairs.
{\bf\large TODO: Compare this to the code! I think that actually, we add throws to
the current set \emph{as we filter}, then merge the set, so a throw in a
\texttt{catch}
body could potentially be filtered out by a subsequent \texttt{catch} associated with the
same \texttt{try}, which is wrong.}

Method calls are handled in a similar manner, as they are potential sources of
exceptions.  A separate try-stack exists for method calls.  When a call is
encountered, it is added to the set with its condition as for throws, but the
container also has a list of exception types filtered by \texttt{catch} clauses.  When
initially added to the top of the method try-stack, the list is empty.  When
processing the \texttt{catch} clauses associated with a \texttt{try}, the list of each
method call instance in the topmost set of the method try-stack is populated. 

In the second phase, the tree is traversed again, and the mappings from the first
phase is used.  When visiting a method definition, any explicit throws from
methods the current node calls are filtered based on the list of \texttt{catch} clauses
enclosing the calls.  Those throws are then added to the set of uncaught
exceptions from the current method, and the whole group along with their
conditions (either complex, via a method call, or the simple condition under
which the exception is thrown) are serialized and added to that method as an
annotation.  When serializing the conditions for explicit throws of a method
call, any formal parameters of the current method used as actual parameters to
callees replace the corresponding callee formals in the serialized condition.
These annotations are compiled into bytecode, and extracted in a later phase of
JavaGrok.

\subsubsection{Precision}
{\LARGE: Discuss funkiness with throws in catch clauses.}

We found our inferred exception conditions to be of sufficient detail.  While
additional information could be gathered by a more complete inter-procedural
propagation, we found our inference able to generate sufficient information for
our evaluation library.  Many of the library methods were simple wrappers to
fill out additional arguments for another method, or made use of utility
argument validators, such as in Figure \ref{fig:argvalidate}.

\begin{figure}
\begin{verbatim}
...
public void count(int x) {
    checkCountTarget(x);
    ...
}
public void checkCountTarget(int y) {
    if (y < 0)
        throw new IllegalArgumentException();
}
\end{verbatim}
\caption{An example of a use of an argument validation method.
\texttt{count(int x)} uses \texttt{checkCountTarget(int y)} to verify that the
provided argument is valid.  Our exception inference detects this and annotates
\texttt{indexList} to indicate that passing a negative number will result in an
\texttt{InvalidArgumentException}.}
\label{fig:argvalidate}
\end{figure}

Buse and Weimer infer documentation for exception conditions
using an analysis that performs full symbolic execution along paths to reach throw
statements, propagating information between methods as necessary~\cite{autodoc}.
Because they
perform full symbolic execution, their tool can be much more precise than ours.
Replacing our exception analysis with theirs, or implementing our own symbolic
execution would likely improve the quality of our exception documentation.  

As in Buse and Weimer's work, we expose internal field names and
local variable names in exceptional conditions.  We agree with their assessment
that while this leaks some
implementation information and is not necessarily informative, in practice
leaked variables are often still useful: for example, showing that an exception is
thrown from a list's \texttt{remove()} method when a local variable \texttt{len}
is equal to 0 has clear implications.

\subsection{Nullability}
\label{sec:Nullability}

In heavily object oriented languages like Java, every variable is nullable,
capable of holding either a reference or the
special value null.  Because of this ubiquity, one common mistake is to call
methods with null parameters they are unprepared to handle, or to erroneously
assume that returned values are never null.  Well documented libraries, like
the Java collections framework, frequently spell out when variables are
assumed to be nullable or not-null, while many libraries with weaker
documentation omit this information entirely.

By means of a nullability type inference~\cite{NonNullTypeInference} we can add
similarly useful information to documentation.  Although the JastAdd analysis
we've chosen to use does not require a whole program, it does make a closed
world assumption; that is, it assumes it is analyzing all code that will ever be
linked with the library.  This leads to optimistic annotation inferences when run on
library code.  JastAdd produces \texttt{@NonNull} annotations, indicating that the user should assume that an
argument or method return value will not assume the value \texttt{null}.
