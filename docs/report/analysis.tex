\section{Analysis}

Our tool will annotate methods with the information if the method returns a 
unique reference (that means a reference to an object that is not referenced 
by any other reference). We think that this information is helpful to the 
programmer because we think it is curcial to now if one is the sole owner of 
an object or not. For a example considering a list object. If one is the sole
owner of the list then one can be sure that after adding an element to the list
and calling a method somewhere in a library the list is guaranteed to still 
contain exactly one elment, because the list was not reachable from anywhere
else.

Furthermore JavaGrok also documents parameters and tells the client
of those functions if a previously unique reference is still unique after the 
called method to which the object was passed as an argument. Analogously to 
above, this information is useful to find out if one is still the sole owner
when one passes a unique reference to a method. Based on our own experience
as programmers we think that knowing such facts helps preventing bugs.