\section{Analysis}

In this section we explain what analysis techniques we are combining, what
output we are using and why we are thinking those are useful for a programmer.

\subsection{Javarifier}
\label{sec:Javarifier}

Javarifier is a tool to infer reference immutability information built in
conjuction with research done by Quinonez et al.~\cite{Javarifier}. It infers
mutability constraints for object fields, method arguments and receivers. Those
constraints may be \texttt{mutable}, \texttt{readonly}, \texttt{?~readonly},
\texttt{polyread}, and \texttt{this-mutable}. The \texttt{?~readonly}
constraint indicates a method argument with a \texttt{readonly} upper bound and
a \texttt{mutable} lower bound. The \texttt{polyread} constraint provides
polymorphism over mutability for method arguments and receivers
(i.e. indicating that a method is read-only when called through a read-only
reference or mutable when called through a mutable reference). A
\texttt{this-mutable} reference provides similar polymorphism for object
fields: a \texttt{this-mutable} field is mutable if \texttt{this} is mutable
and read-only otherwise.

Javarifier produces its results directly in the JAIF format which allows us to
easily integrate it into our toolchain. We need simply to express the meaning
of its inferred constraints in the Javadoc documentation in concise language.

\subsection{Uno}

Uno is an open source tool which is the outcome of research done by Ma and
Foster~\cite{Uno}. It infers alias and encapsulation properties for Java.  The
tool generates properties which provide information about how a certain
function treats its parameters, return values and fields when called. 
E.g. if a function captures or leaks a reference or returns an unique 
reference (that means a reference to an object that is not referenced 
by any other reference).

The tool generates annotations which are stored in a single separate file. 
To integrate the analysis results of UNO into the Java documentation we
use our own framework. We implemented a tree visitor inside the Java 7 
compiler that, at initialization, reads in the file generated by UNO and 
stores the information in a hashset. During the iteration over the AST
our visitor inserts annotation at the appropriate places.

Our tool will annotate methods with the information if the method returns a 
unique reference or not. The two annoations we add to the source code are called
@UniqueReturn and @NonUniqueReturn. We think that this information is helpful to the 
programmer because we think it is curcial to now if one is the sole owner of 
an object or not. For an example considering a list object. If one is the sole
owner of the list then one can be sure that after adding an element to the list
and calling a method somewhere in a library the list is guaranteed to still 
contain exactly one elment, because the list was not reachable from anywhere
else.

Furthermore JavaGrok also documents parameters and tells the client
of those functions if a previously unique reference is still unique after the 
called method to which the object was passed as an argument. Analogously to 
above, this information is useful to find out if one is still the sole owner
when one passes a unique reference to a method. Based on our own experience
as programmers we think that knowing such facts helps preventing bugs.

\subsection{Thrown Exceptions}

Java requires that checked exceptions be declared by methods that raise them,
but it unfortunately cannot require that they be well documented, and
frequently the conditions under which exceptions are raised are unclear to a
library user. Additionally, unchecked exceptions need not be declared in method
signatures nor documentation and can remain entirely invisible to the library
user until they are encountered at runtime.

Buse and Weimer infer documentation for the conditions that cause some
exceptions~\cite{autodoc}.  Their analysis locates explicit throws of
exceptions and performs some symbolic execution along with propagating results
between methods to determine exceptional input conditions.  We do not use an
existing analysis tool for this purpose, as none were available.  Our analysis
is a fairly simple search for explicit throws, tracking branch statements on the
way to such statements and propagating some information about thrown runtime
(unchecked) exceptions between methods.

\subsection{Nullability}
\label{sec:Nullability}

In heavily object oriented languages like java, every variable is nullable
(aka. option, maybe), capable of holding either a reference to a value or the
special value null.  Because of this ubiquity, one common mistake is to call
methods with null parameters they are unprepared to handle, or to erroneously
assume that returned values are never null.  Well documented libraries, like
the Java collections framework, will frequently spell out when variables are
assumed to be nullable or not-null when an ambiguity seems likely.

By means of a nullability inference~\cite{NIT,NonNullTypeInference} we can add
similarly useful type decorations to documentation.  Although two Java
nullability inferences are freely available, they are both whole program
analyses and therefore ill-suited to our target application: library
documentation.  We plan to first try adapting their existing analysis code, or
failing that re-implement a modular version of their analysis within our own
framework.
