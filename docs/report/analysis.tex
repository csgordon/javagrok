\section{Analyses}

In this section we describe the analyses we perform: their implementation,
output, and why each analysis would be useful for a developer.

\subsection{Javarifier}
\label{sec:Javarifier}

Javarifier is a tool to infer reference immutability information built in
conjuction with research done by Quinonez et al.~\cite{Javarifier}. It infers
mutability constraints for object fields, method arguments and receivers.  We
use \textit{mutate} to mean updating the value of an object's fields, calling a
method on an object that updates the values of its fields, or calling a method
on an object that mutates an object referenced by one of its fields.

Below we enumerate the annotations inferred by Javarifier which are utilized by
JavaGrok, and a summary of what they communicate to the developer:

\texttt{@ReadOnly} When appearing on a method, this indicates to the developer
that the method will not mutate its receiver. When appearing on a method
argument, this indicates to the developer that the method will not mutate the
argument through the supplied reference.

\texttt{@Mutable} When appearing on a method, this indicates to the developer
that the method may mutate its receiver. When appearing on a method argument,
this indicates to the developer that the method may mutate the argument through
the supplied reference.

Javarifier generates two additional annotations which are important for type
checking, but not important when simply communicating method behavior to a
developer.

\texttt{@QReadOnly} appears on type parameters of method arguments, for example
\texttt{List<@QReadOnly Date>}. Such an annotation on a method argument
indicates that the method accepts both \texttt{List<@ReadOnly Date>} and
\texttt{List<@Mutable Date>}. However, the method itself is restricted to only
operations allowed on both read-only and mutable dates, which is strictly a
subset of the operations allowed on read-only dates.

\texttt{@PolyRead} indicates that a method and/or its arguments are polymorphic
over mutability. Such a method may legally be instantiated with
\texttt{@ReadOnly} receiver and/or arguments and thus must not mutate the
receiver or arguments.

We convert both of the above annotations to \texttt{@ReadOnly}, as
that is sufficient to communicate the method's behavior to the developer and we
are not concerned with type-checking mutability.

\subsection{Uno}

Uno infers alias and encapsulation properties.  The
tool generates annotations which provide information about how a certain
function treats its parameters, return values and fields when called. More
concretely, it infers whether or not a function captures or leaks a
reference, or returns a new unique reference.

The tool generates annotations which are stored in a single separate file. 
To integrate the analysis results of UNO into the Java documentation we
use our own framework. We implemented an AST traversal inside the Java 7 
compiler that, at initialization, reads in the file generated by UNO and 
stores the information in a hashset. During the iteration over the AST
our visitor inserts annotation at the appropriate places.

\subsection{Thrown Exceptions}

Java requires that checked exceptions be declared by methods that raise them,
but it unfortunately cannot require that they be well documented, and
frequently the conditions under which exceptions are raised are unclear to a
library user. Even with checked exceptions, often a supertype of a thrown
exception is declared.  While knowing that an \texttt{IOException} might be
thrown is useful, knowing that the method might throw either a
\texttt{FileNotFoundException} or \texttt{InterruptedByTimeoutException} is more
useful to a developer, because knowing the specific exception may help to debug
a program.  Additionally, unchecked exceptions need not be declared in Java method
signatures or documentation and can remain entirely invisible to the library
user until they are encountered at runtime.

To alleviate this, we infer some of the conditions under which a method would
throw an exception, and produce annotations of the form:

\begin{verbatim}
@ExceptionProperty(throwsWhen =
    "IllegalArgumentException when (x < 0)")
public Object getElement(int x) { ... }
\end{verbatim}

This sort of inferred information can inform developers of unchecked exceptions,
and alert them to otherwise undocumented assumptions of a library without
requiring developers to dig through the library's code.
Annotations may contain a list of exceptions and conditions, and for some
complex conditions we reduce the condition to ``sometimes'' because our analysis
is purely syntactic.  We do not use an
existing analysis tool because none were available for this task, but even our
syntactic analysis provides useful results.

Our analysis is a fairly simple search for explicit throws, tracking branch
statements at the purely syntactic level on the
way to such statements.  We propagate information about exceptional conditions
between methods by purely syntactic replacement of formal arguments by actual
arguments.  We do not account for intermediate side effects on variables, or
track data flow from arguments to local variables or object fields.  In
practice, we have not found this lack of precision to be problematic as most of
the thrown exceptions we see are from argument validation.

Buse and Weimer infer documentation for exception conditions
using an analysis performs that full symbolic execution along paths to reach throw
statements, propagating information between methods as necessary~\cite{autodoc}.
Because they
perform full symbolic execution, their tool can be much more precise than ours.
Replacing our exception analysis with theirs, or implementing our own symbolic
execution would likely improve the quality of our exception documentation.  

As in Buse and Weimer's work, we expose internal field names and
local variable names in exceptional conditions.  We agree with their assessment
that while this leaks some
implementation information and is not necessarily informative, in practice
leaked variables are often still useful: for example, showing that an exception is
thrown from a list's \texttt{remove()} method when a local variable \texttt{len}
is equal to 0 has clear implications.

\subsection{Nullability}
\label{sec:Nullability}

In heavily object oriented languages like Java, every variable is nullable,
capable of holding either a reference or the
special value null.  Because of this ubiquity, one common mistake is to call
methods with null parameters they are unprepared to handle, or to erroneously
assume that returned values are never null.  Well documented libraries, like
the Java collections framework, frequently spell out when variables are
assumed to be nullable or not-null, while many libraries with weaker
documentation omit this information entirely.

By means of a nullability type inference~\cite{NonNullTypeInference} we can add
similarly useful information to documentation.  Although the JastAdd analysis
we've chosen to use does not require a whole program, it does make a closed
world assumption; that is, it assumes it is analyzing all code that will ever be
linked with the library.  This leads to optimistic annotation inferences when run on
library code.  JastAdd produces \texttt{@NonNull} annotations, indicating that the user should assume that an
argument or method return value will not assume the value \texttt{null}.
