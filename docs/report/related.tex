\section{Related Work}
There is a wealth of work on inferring interesting properties of existing code,
but most of these are focused on inferring properties suitable for checking.
Because of this, few projects have performed user studies to evaluate how
inferred information may assist developers.

The most directly relevant work is Buse and Weimer's system for automatically
inferring documentation for conditions that will result in a Java method
throwing an exception~\cite{autodoc}.  We implement a similar analysis
to detect the conditions under which some exceptions are thrown, though weaker
because we do not perform full symbolic execution.  They also performed a
comparison of generated documentation to exemplar documentation for several
moderately-sized Java projects.  We are the first to perform a user study of how
helpful to programmers these inferred properties can be when used to augment existing
documentation.

Analysis of libraries as a separate entity from whole programs, without
considering documentation, has been considered by the software
engineering community.  Arnout and Meyer performed a study of the .NET
collections libraries where they manually extracted
contracts~\cite{findingcontracts}.  Their work is complementary to ours: in some
sense we are extracting and documenting the implicit contracts of libraries,
while they extract the implicit contracts and present them for static checking.

Williams et al.~describe a method for detecting possible deadlocks in Java
libraries~\cite{deadlocklibs}.  Sankaranarayanan et al.~use inductive logic
programming to infer precise Datalog specifications of proper use of library
APIs~\cite{mininglibspecs}.  Their work is intended for consumption by the
developer, but they do not go so far as to modify whatever documentation a
library has.  Either of these approaches, or others, could likely be adapted to
generate additional useful documentation.

Tools for analyzing documentation, particularly in relation to program text,
have also received some attention.  Schreck, Dallmeier, and Zimmermann describe
a tool for measuring the quality of software documentation, and how that
quality varies over time~\cite{evolvedoc}. Tools like Mismar~\cite{mismar} make
it easier to construct usage guides for code bases by allowing the developer to
essentially drag-and-drop software components to form a documentation skeleton.
eMoose~\cite{emoose} performs a programming-environment version of
inter-procedural constraint propagation; it integrates
documentation about a method being called into the hover text for a programmer
working at a call site for that method, and they performed a user study to show
that the new presentation of documentation helped developers.  Any of these techniques could be used
in conjunction with our work to further improve the developer experience.

\subsection{Individual Analyses}

There has been some recent work in nullability analysis for Java
resulting in two available tools.  Hubert~\cite{NIT} implemented a nullability
analysis for Java bytecode based on abstract interpretation techniques.  This
analysis is instantiated in the {\sc NIT} tool.  Ekman and
Hedin~\cite{NonNullTypeInference} independently implemented a nullability type
inference analysis.  Both of these tools perform whole program analyses, and
consequently require modification to fit into our modular analysis framework.

Quinonez, et al.~\cite{Javarifier} described a technique for inference of
reference immutability in Java and implemented it in a tool called {\sc
  Javarifier}. Their goal is the inference of annotations for the {\sc Javari}
language (an extension of Java) which enforces reference immutability
constraints. TODO: IGJ~\cite{IGJ}, Pidasa~\cite{Pidasa}.

Cherem and Rugina~\cite{UniquenessInference} described a uniqueness inference
technique using a two-level abstraction: a combination of a flow-sensitive
intraprocedural analysis with a flow-insensitive interprocedural analysis. 
They can infer uniqueness information which is used to actively free Java
object whenever an unique reference is lost.  Ma and
Foster~\cite{Uno} describe their inference tool Uno that uses the same
algorithmic idea. They infer a variety of alias and encapsulation
properties, including uniqueness, lending and aliasing.  Aldrich,
Kostadinov and Chambers~\cite{AliasJava} showed how alias information helps the
programmer to understand how data is shared in large software systems.
They go beyond then just use the alias information for documentation purposes:
They define type checking rules for their system. Hence their analysis has to be
sound. In contrast we can also accept unsound annotations if they are still helpful for the 
programmer's understanding of the code. None of the above
have performed a user study to see how the inferred information can help a
programmer to understand unsufficently documented code.

Buse and Weimer~\cite{autodoc} automatically infer documentation for
exceptions.  Our exception inference is weaker than theirs due to the lack of
symbolic execution in our system.  One
possible future extension of our work would be to integrate their exception
analysis.  They evaluated their tool through a manual comparison of inferred documentation to
existing documentation for several moderately-sized Java projects considered to
have decent documentation.  We performed a live user study using a library with what
we felt to be average documentation, comparing the progress and feedback of
developers using existing documentation to that of developers using augmented
documentation.
