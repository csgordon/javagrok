\section{Related Work}
There is a wealth of work on inferring properties of existing code,
but most of these are focused on inferring properties suitable for automatic
verification.
Because of this, few projects have performed user studies to evaluate how
inferred information may assist developers.

The most directly relevant work is Buse and Weimer's system for automatically
inferring documentation for conditions that will result in a Java method
throwing an exception~\cite{autodoc}.  We implement a similar analysis
to detect the conditions under which some exceptions are thrown, though weaker
because we do not perform full symbolic execution.  They also performed a
comparison of generated documentation to exemplar documentation for several
moderately-sized Java projects.  We are the first to perform a user study
examining how
helpful these inferred properties can be for programmers when used to augment existing
documentation.

Analysis of libraries as a separate entity from whole programs, without
considering documentation, has been considered by the software
engineering community.  Arnout and Meyer performed a study of the .NET
collections libraries where they manually extracted
contracts~\cite{findingcontracts}.  Their work is complementary to ours: in some
sense we are extracting and documenting the implicit contracts of libraries,
while they extract the implicit contracts and present them for static checking.

Williams et al.~describe a method for detecting possible deadlocks in Java
libraries~\cite{deadlocklibs}.  They evaluated their tool quantitatively in
terms of how many potential deadlocks they found and were able to confirm by
hand in several libraries, and on how well they could control the number of
false positives.

Sankaranarayanan et al.~use inductive logic
programming to infer precise Datalog specifications of proper use of library
APIs~\cite{mininglibspecs}.  Their work is intended for consumption by the
developer, but they do not go so far as to modify whatever documentation a
library has.  Either of these approaches, or others, could likely be adapted to
generate additional useful documentation.

Tools for analyzing documentation, particularly in relation to program text,
have also received some attention.  Schreck, Dallmeier, and Zimmermann describe
a tool for measuring the quality of software documentation, and how that
quality varies over time~\cite{evolvedoc}. They evaluate their tool by running
it against several versions of the Eclipse source code, and look at how the
score assigned by their tool varied over time.  Tools like Mismar~\cite{mismar} make
it easier to construct usage guides for code bases by allowing the developer to
essentially drag-and-drop software components to form a documentation skeleton.
We could find no details on Mismar's evaluation.
eMoose~\cite{emoose} performs a programming-environment version of
inter-procedural constraint propagation; it integrates
documentation about a method being called into the hover text for a programmer
working at a call site for that method, and they performed a user study to show
that the new presentation of documentation helped developers.  Their work
focused on how presentation of documentation can help developers, while our work
focuses on how we can automatically augment documentation to contain more
information.  Any of these
techniques could probably be used
in conjunction with our work to further improve the developer experience.

\subsection{Individual Analyses}

There has been a lot of work on nullability analysis for Java.
Engelen provides a useful (albeit slightly dated) survey in his master's thesis~\cite{Engelen2006}.
Hubert~\cite{NIT} implemented a nullability
analysis for Java byte-code based on abstract interpretation
techniques, producing the {\sc NIT} tool.  Fausto Spoto~\cite{Spoto2008} has also developed
an abstract interpretation based analysis in his Julia tool.
Ekman and Hedin~\cite{NonNullTypeInference} implemented a nullability type
inference analysis in JastAdd.  Ultimately we chose to use 
Ekman and Hedin's JastAdd inferencer because it's a static analysis, does not require whole program analysis and is readily available for
integration into JavaGrok.  None of these authors evaluated whether their
nullability annotations were beneficial to users, only whether they produced
accurate annotations.

% Quinonez, et al.~\cite{Javarifier} described a technique for inference of
% reference immutability in Java and implemented it in a tool called {\sc
%   Javarifier}. Their goal is the inference of annotations for the {\sc Javari}
% language (an extension of Java) which enforces reference immutability
% constraints. %% TODO: IGJ~\cite{IGJ}, Pidasa~\cite{Pidasa}.

Cherem and Rugina~\cite{UniquenessInference} described a uniqueness inference
technique using a two-level abstraction: a combination of a flow-sensitive,
intraprocedural analysis with a flow-insensitive interprocedural analysis. 
They can infer uniqueness information which is used to actively free Java
objects whenever an unique reference is lost. As they focus on optimizing 
early deallocation they do not evaluate their inference in respect to usefulness
for a programmer to understand an API.

Ma and Foster~\cite{Uno} describe their inference tool Uno, which uses a similar
algorithm. They infer a variety of alias and encapsulation
properties, including uniqueness, lending and aliasing. In their evaluation
they focus on statistics: for example how many times Uno could infer a certain property.
But they do not evaluate how their inferred properties can be used, either by a tool
or by the user.

Aldrich, Kostadinov and Chambers~\cite{AliasJava} show alias information 
can be used at compile time to check if the alias annotations are actually
preserved. Their hypothesis is that this additional safety and annotations will
help the programmer to understand the program. In their paper they mainly focus
on the alias type system and on how it's built up. But they also see the need
for an automatic inference and therefore they also describe an inference algorithm.
They perform two evaluations on real projects where a programmer
annotated the code by hand with alias information. They also test their
alias annotation inference algorithm, but they conclude that
their inference tool inferred too many annotations and was too precise to be actually useful
for developers.

Buse and Weimer~\cite{autodoc} automatically infer documentation for
exceptions.  In addition to differences in the evaluation of our work, they use
an analysis that performs symbolic execution along paths to reach throw
statements, propagating information between methods as necessary.
Because they
perform symbolic execution, their tool can be more precise than ours, though
they also lose precision by ignoring loop iteration.
Replacing our exception analysis with theirs, or implementing our own symbolic
execution would likely improve the quality of our exception documentation.
