\section{Example}

Here is an example what our tool is capable of doing.
Note that by just looking at the method interface of 
\texttt{public Object getAll()} a programmer does not know
that \texttt{getAll()} will leak it's reference to it's
internal representation. This might be a big surprise
if the list of objects that get's returned keeps growing
even though the caller of \texttt{getAll()} does not add those object.


\begin{lstlisting}
public class A
{
	private LinkedList<Object> list;
		
	public A() {
		list = new LinkedList<Object>();
	}
		
  public void add(Object o) {
   	list.add(o);
  }
    
	public List<Object> getAll() {
		return list;
  }
}
\end{lstlisting}

But with the annotations of our tool a programmer can 
determine that she or he will not have an uniqe reference
to the list of all objects. And might first copy the content 
of the list into an own, newly created list.

\begin{lstlisting}
public class A
{
	private LinkedList<Object> list;
	
	@UniqueReturn
	public A() {
		list = new LinkedList<Object>();
	}
		
  public void add(@Retained Object o) {
   	list.add(o);
  }
  
	@NonUniqueReturn
	public List<Object> getAll() {
		return list;
  }
}
\end{lstlisting}

In the resulting Javadoc html document our inferred annotations
for \texttt{getAll()} will show up like this:\\
\\
{\bf getAll}\\
\\
@NonUniqueReturn\\
public Object getAll()\\