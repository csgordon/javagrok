\section{Example}
\label{sec:Example}

Consider Figure \ref{fig:pre}, a na\"ive collection object.
A programmer looking at the method interface of 
\texttt{public Object getAll()} does not know
that \texttt{getAll()} will leak a reference to its
internal representation. This might be surprising
if the list of objects that gets returned keeps growing
even though the caller of \texttt{getAll()} does not add those objects.

\begin{figure}[h]
\begin{lstlisting}
public class A
{
	private LinkedList<Object> list;
		
	public A() {
		list = new LinkedList<Object>();
	}
		
  public void add(Object o) {
   	list.add(o);
  }
    
	public List<Object> getAll() {
		return list;
  }
}
\end{lstlisting}
\label{fig:pre}
\caption{An example unannotated class.}
\end{figure}

But with information about reference leaks and caputures added, a programmer can 
determine that she or he will not have an uniqe reference
to the list of all objects.  Knowing the reference is still shared by the
library, a developer may decide to copy the content 
of the list into a newly created list.  Figure \ref{fig:postannotation}
shows a class annotated with reference leak and capture information produced by
our tool.

\begin{figure}[h]
\begin{lstlisting}
public class A
{
	private LinkedList<Object> list;
	
	@UniqueReturn
	public A() {
		list = new LinkedList<Object>();
	}
		
  public void add(@Retained Object o) {
   	list.add(o);
  }
  
	@NonUniqueReturn
	public List<Object> getAll() {
		return list;
  }
}
\end{lstlisting}
\label{fig:postannotation}
\caption{An example class annotated with capture and leak information.}
\end{figure}

For a simple example such as this, inspecting the library code is not a
significant burden for the developer.  But as library size and complexity grow,
it is preferable to refer to condensed documentation rather than
deciphering library code.
Fortunately Javadoc will reproduce any source annotations in the HTML
documentation it produces.  For \texttt{getAll()}, the documentation will
include:\\
\\
{\bf getAll}()\\
\\
@NonUniqueReturn\\
