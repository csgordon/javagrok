\section{Introduction}

Despite their best intentions, many developers routinely fail to provide
adequate or any documentation of the code they write.  Whether because of bad
practice, laziness or a sincere belief on the part of programmers that their
code will soon be thrown away, much code in regular use and reuse remains
undocumented.  New developers join the team, the code gets handed off to
another group, or perhaps even posted publicly on the Internet.  By various
means, this code finds its way into the hands of programmers who---having been
assured that this code will save them weeks of effort---now face a thoroughly
unenviable task: grok a lump of un(der)documented code and figure out its
interface enough to solve their original problem.

Ideally, we would have an army of ingenious, classically trained Shakespearean
typing monkeys, ready to provide high-quality documentation for all our
programs and libraries at the drop of a hat.  However in reality, we, the
unlucky developers often must make do with a single use case or oblique e-mail
offering advice---if we are even that lucky.  Sitting at our desks, we curse
fate, the code in front of us, and the anonymous (or not so anonymous)
programmer responsible for our pain.  We wonder, wouldn't it be nice if we
could just push a button and get some useful, if not perfect documentation?

We designed a tool, named JavaGrok, to do just this: take an un(der)documented
library and opportunistically infer bare-bones documentation to aid a client
programmer.  Our tool takes a collection of java source files intended for use
as a library, computes a variety of modular static analyses, and then places
select results from these analyses into the javadoc documentation for the
provided classes.  These analysis results appear side by side, line by line
with any pre-existing documentation.  They warn the library's user about
potentially null return values, possible side-effects and unhandled exceptions,
to name a few examples.  Armed with this reinforced documentation, our
unfortunate programmer is now ready to do battle with their unseemly library.

Certainly, any attempt to provide automatically generated documentation is
prone to being useless or even actively unhelpful if done incorrectly.  In
order to verify that the provided automatic documentation helps rather than
hurts users, we have conducted a user study.  Programmers were asked to
accomplish a routine task(?) using an unfamiliar library and lacking extensive
or robust documentation.  It turns out we discovered some interesting results.
Go us!
