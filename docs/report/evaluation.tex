\section{Evaluation}
\label{sec:Evaluation}

We performed a small experiment wherein developers were given a third party
library and a set of programming tasks and were asked to report on certain
events that occurred during the use of the library to complete the tasks. We
specifically focused on when the developer had questions about how to correctly
use the library. We considered four cases: a) the question was answered by our
annotations, b) the question was answered by the existing library
documentation, c) the question was answered by direct referal to the library
source code, and d) the question was unable to be answered. We also asked the
developers to note any time they were surprised by the library's behavior.

Our hypothesis is that by augmenting library documentation with properties
inferred via static analyses, we will decrease the frequency with which
developers must refer to the library source code to answer questions. This
should show up as a non-zero incidence of questions being answered by
annotations and a corresponding decrease in the questions that were answered by
source code. We also hypothesized that our annotations may have resulted in a
reduction in surprise at library behavior, and would not increase the incidence
of surprise.

\subsection{Experiment Design}

\begin{figure*}
\centering
\begin{tabular}{ l | r r r r | r }
 & \multicolumn{4}{| c | }{Question answered by:} & \\
Developer & Annotations & Docs & Source & Unanswered & Surprised \\
\hline
Dev 1 & 0 & 14 & 0 & 1 & 0 \\
Dev 2 & 1 &  5 & 2 & 1 & 0 \\
Dev 3 & 1 &  3 & 0 & 0 & 0 \\
Dev 4 & 1 &  6 & 1 & 1 & 0 \\
\hline
\textit{Experiment} & 3 & 28 & 3 & 3 & 0 \\
\hline
Dev 5 & - &  4 & 0 & 0 & 0 \\
Dev 6 & - & 13 & 2 & 3 & 1 \\
Dev 7 & - & 10 & 0 & 2 & 0 \\
Dev 8 & - &  7 & 1 & 2 & 1 \\
\hline
\textit{Control} & - & 34 & 3 & 7 & 2 \\
\hline
\end{tabular}
\caption{Experiment Results}
\label{fig:exp_results}
\end{figure*}

The experiment involved two groups of four developers each. The developers were
given a Java library with which they are unfamiliar and a set of programming
tasks to be implemented using the functionality provided by the library. One
group was provided with augmented documentation and the other provided with
original, unmodified documentation.

The members of each group were provided with equivalent development
environments (the Eclipse IDE, configured to provide ready access to the
augmented or unmodified documentation as well as the augmented or unmodified
library source code). As the subjects' experience with the Eclipse IDE was
quite varied, they were shown how to access both the library documentation and
source code. The aim was to reduce the likelihood that differing familiarity
with the IDE would influence their decision to inspect the documentation or
source code.

The subjects were instructed to first check the documentation when they
encountered a question about correct usage of the library, then to consult the
source code only if their question was not answered to their satisfaction by
the documentation. Finally they were asked to record the outcome of each event.
The experimenter could record a question as being resolved by ``annotations'',
``documentation'', ``source code'' and ``not resolved''. The control group had
only the latter three choices as they had unaugmented documentation.

We also instructed subjects to complete a short survey after completing the
experimental tasks. The results of this survey were not intended to directly
validate or refute our hypothesis, but to help us to gauge other aspects of the
work, better understand ambiguous results, and to direct future work. Elements
from the survey are mentioned in the discussion.

The results of our experiment are shown in Figure ~\ref{fig:exp_results} and
discussed below.

\subsection{Discussion}
Overall, our results were inconclusive.  Our sample size of developers is not
large enough to be statistically significant, but the developer surveys exhibit
a mixture of both positive and negative trends.

Most developer complaints from both the experimental and control groups were
about the high-level usage model of the library being unclear.  This suggests the task we chose was
not good for evaluating our tool; our tool infers small technical
properties of the code, not use models for the library.  A better task would
have required use of our annotations or close source inspection to avoid some
subtle bugs.  Only the control group ever said they were surprised by library
behavior, but the surprises were about higher-level component interactions such
as what order to call a pair of methods.

None of the developers in the experimental group found the annotations to be
frequently useful.  None answered more than one question through the
annotations, which was never more than 1/9th of the questions they had.  Most
questions for both groups were answered by reading the existing documentation.

There was some positive feedback from the experimental group, however.  All
found the annotations to be of a useful level of detail, and 3 of 4 found them
to be potentially useful, but not for the evaluation task.  This generally
positive response reinforces our belief that the properties
JavaGrok infers could be useful.  It is also possible that other properties
exist that could have been inferred and would have been useful for our chosen
evaluation task.

In designing
our experimental task, we struggled to find a balance between designing a
realistic task, and designing a task geared towards requiring annotations.  We
settled on a task we felt was reasonably general and might benefit from the
generated annotations.  Designing a task specifically to make the annotations useful would
not have helped us understand how useful they were generally, but the situations
when they would provide benefit seem not appear to occur in such a short exercise.
Upon reflection, we suspect that the target experiment time of one hour per
subject, chosen to encourage volunteers, may simply not be enough time to run
into tricky cases where the very particular information JavaGrok documents would
be useful.  We suspect that the annotations might prove valuable in a longer
study, using a library over a longer period of time, allowing more time to run
into what we acknowledge are usually corner cases where specific information
about JavaGrok's properties would prove useful.  One of our subjects said he
suspects they may prove more useful in debugging than in development.

While our actual results showed no significant benefit to JavaGrok's annotations
for this task, the generally positive response to the idea from our test
subjects suggests that automatically inferred documentation deserves more
thorough study.

