\section{Evaluation} \label{sec:Evaluation}
We performed a small experiment wherein developers were given a third party
library with documentation and a set of programming tasks. They were asked to report on certain
events that occurred while completing the tasks. We specifically focused on
when the developer had questions about how to correctly use the library, and
considered four cases: a) the question was answered by our annotations, b) the
question was answered by the existing library documentation, c) the question
was answered by reading the library source code, and d) the question was not
answered. We also asked the developers to note any time they were surprised by
the library's behavior.

Our hypothesis was that by augmenting library documentation with inferred
properties, we would decrease the frequency with which developers needed to
refer to the library source code to answer questions. This would show up as a
non-zero incidence of questions being answered by annotations and a
corresponding decrease in the questions that were answered by reading source
code. We also hypothesized that our annotations may have resulted in a
reduction in surprise at library behavior, and would not increase the incidence
of surprise.

\begin{figure*}
\centering
\begin{tabular}{ l | r r r r | r }
 & \multicolumn{4}{| c | }{Question answered by:} & \\
Developer & Annotations & Docs & Source & Unanswered & Surprised \\
\hline
Dev 1 & 0 & 14 & 0 & 1 & 0 \\
Dev 2 & 1 &  5 & 2 & 1 & 0 \\
Dev 3 & 1 &  3 & 0 & 0 & 0 \\
Dev 4 & 1 &  6 & 1 & 1 & 0 \\
\hline
\textit{Experiment} & 3 & 28 & 3 & 3 & 0 \\
\hline
Dev 5 & - &  4 & 0 & 0 & 0 \\
Dev 6 & - & 13 & 2 & 3 & 1 \\
Dev 7 & - & 10 & 0 & 2 & 0 \\
Dev 8 & - &  7 & 1 & 2 & 1 \\
\hline
\textit{Control} & - & 34 & 3 & 7 & 2 \\
\hline
\end{tabular}
\caption{Experiment Results}
\label{fig:exp_results}
\end{figure*}

\subsection{Experiment Design}
The experiment involved two groups of four developers each, all of whom were
University of Washington computer science graduate students. The developers
were given a set of programming tasks involving the creation of simple
interactive animations. They were supplied with the Nenya~\cite{nenya} graphics
and animation library to use to complete those taks. None of the developers had
previously seen or used the library. The experimental group was provided with
augmented documentation and source code for the library, and the control group
was provided with original, unmodified documentation and source code.

There were four tasks that were designed to consume at least one
hour. Developers were told that they were free to leave after one hour, even if
they had not completed all of the tasks.

The members of each group were provided with the Eclipse IDE, configured to
provide ready access to the appropriate documentation and source code. As the
subjects' experience with the Eclipse IDE was quite varied, they were shown how
to access both the library documentation and source code. The aim was to reduce
the likelihood that differing familiarity with the IDE would influence their
decision to inspect the documentation or source.

The subjects were instructed to first check the documentation when they
encountered a question about correct usage of the library, then to consult the
source code only if their question was not answered to their satisfaction by
the documentation. Finally they were asked to record the outcome of each event.
The experimenter could record a question as being resolved by ``annotations'',
``documentation'', ``source code'' and ``not resolved''. The control group had
only the latter three choices as they had unaugmented documentation.

We also instructed subjects to complete a short survey after finishing the
experimental tasks. The results of this survey were not intended to directly
validate or refute our hypothesis, but to help us to gauge other aspects of the
work, better understand ambiguous results, and to direct future work. Data from
the survey are mentioned in the discussion below. The results of our experiment
are shown in Figure ~\ref{fig:exp_results}.

\subsection{Discussion}
For our small test group, a striking difference between the control and
experimental groups would have been necesary to support our hypothesis.  The
results had no such strong
trends, rather they were inconclusive.  Our sample size of developers was not
large enough to be statistically significant, but the developer surveys exhibit
a mixture of both positive and negative trends.

Most developer complaints from both the experimental and control groups were
about the high-level usage model of the library being unclear.  This suggests
the evaluation task was
a poor choice for evaluating our tool; our tool infers small technical
properties of the code, not use models for the library.  A better task would
have required use of our annotations or close source inspection to avoid some
subtle bugs.  Only the control group ever said they were surprised by library
behavior, but the surprises were about higher-level component interactions such
as what order to call a pair of methods, not about properties JavaGrok can
infer.

None of the developers in the experimental group found the annotations to be
frequently useful.  None answered more than one question using the
annotations, which was never more than 1/9th of a subject's questions.  Most
questions for both groups were answered by reading the existing documentation.

There was however some positive feedback from the experimental group.  All
found the annotations to be of a useful level of detail, and 3 of 4 said they
considered the annotations
to be potentially useful, but not for the evaluation task.  This generally
positive response reinforces our belief that the properties
JavaGrok infers could be useful.  It is also possible that other properties
exist that could have been inferred and would have been useful for our chosen
evaluation task.

In designing
our experimental task, we struggled to find a balance between designing a
realistic task, and designing a task geared towards requiring annotations.  We
settled on a task we felt was reasonably general and might benefit from the
generated annotations.  Designing a task specifically to make the annotations useful would
not have helped us understand how useful they were generally, but the situations
where they would provide benefit seem not appear to occur in such a short exercise.
Upon reflection, we suspect that the target experiment time of one hour per
subject, chosen to encourage volunteers, may simply not be enough time to run
into tricky cases where the very particular information JavaGrok documents would
be useful.  We suspect that the annotations might prove valuable in a longer
study, using a library over a longer period of time, allowing more time to run
into what we acknowledge are usually corner cases where specific information
about JavaGrok's properties would prove useful.  One of our subjects said he
suspects they may prove more useful in debugging than in development.

While our actual results showed no significant benefit to JavaGrok's annotations
for this task, the generally positive response to the idea from our test
subjects suggests that automatically inferred documentation deserves more
thorough study.

