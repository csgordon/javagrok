\section{Evaluation}

We propose to evaluate our work in two ways: by comparing our inferred
documentation against existing documentation for a library that is widely
considered to be of high quality (\textit{exemplar} documentation), and by
conducting a small controlled experiment.

\subsection{Comparison with exemplar documentation}

We will generate documentation by running our our tool on a subset of the
classes in the \texttt{java.util} package of the Java standard libraries. This
documentation will be compared with the documentation provided with the
Java\texttrademark SE Development Kit version 1.6~\cite{JDK6}.

For each class and method in the target code, our tool generates properties for
insertion into the documentation. Example properties include: ``Argument foo
may not be null'', or ``This method has no side effects.'' We will compare
these properties to properties identified in the exemplar documentation and
classify them via the following criteria:

\begin{itemize}
\item How many properties were inferred by our tool that were not specified by
  the exemplar documentation.
\item How many properties were specified by the exemplar documentation that
  were not inferred by our tool.
\item In cases where our tool documents a property that is also specified by
  the exemplar documentation, is our documentation \textit{better},
  \textit{worse} or about the \textit{same}. We will consider our documentation
  better if it is more precise, or documents additional conditions for the
  property. We will consider it the same if it conveys the same information as
  the exemplar documentation. We will consider it worse in all other cases.
\end{itemize}

We will specifically consider the frequency with which our tool generates the
same or better documentation than the exemplar. In that case our tool is at a
minimum saving the developer the time it takes to write the documentation and
potentially also improving its quality.

We will also consider the frequency with which our tool fails to document
properties expressed in the exemplar documentation or generates properties
classified as worse. In these cases our tool is failing to save the developer
time, or actively worsening the quality of the documentation.

We will also consider the possibility that our tool generates so many
properties that it overwhelms the reader, though this more subjective measure
will be evaluated as a part of the controlled experiment.

\subsection{Controlled experiment}

We will perform a controlled experiment where two groups of developers are
given a third party library (including its source code) and a skeleton program
and are asked to implement functionality in the program that makes use of the
library. One group will be provided with the original documentation supplied
with the library and the other group will be supplied with documentation
augmented by our tool.

TODO: describe exactly which library and what programming tasks will be
required.

We will provide a checklist on which the developers will be asked to
self-report the following events:

\begin{itemize}
\item The number of times developers referred to the library API documentation.
\item The number of times developers refereed to the library source code.
\item The number of times developers experienced unexpected behavior from the
  library. For example, encountering a null pointer exception because they
  supplied null as an argument to a library method that they thought would
  handle nulls.
\end{itemize}

We will also provide them with instructions on how to access the library
documentation as well as the library source code to insure that they are not
otherwise discouraged from referencing either.

We will also record the total time taken to complete the tasks and manually
inspect the resulting programs to tabulate the number of errors made in using
the underlying library. For example, we will look for situations where the
program could pass invalid data to the library or where the program fails to
call proper library initialization or cleanup methods.

Finally, we will conduct an exit survey, asking the developers subjective
questions about their experiences:

\begin{itemize}
\item Did you find the documentation to be too detailed, sufficiently detailed
  or not detailed enough?
\item Did you find the experience of using this library to be pleasant, normal
  or frustrating.
\end{itemize}

We will compare the data from the control group with those from the
experimental group in the following ways:

\begin{itemize}
\item Did the experimental group refer to the library source code less
  frequently?
\item Were there fewer instances of unexpected library behavior encountered by
  the experimental group?
\item TBD...
\end{itemize}
